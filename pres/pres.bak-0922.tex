\documentclass[10pt]{beamer}

\mode<presentation>
{
  % theme: https://github.com/matze/mtheme
  \usetheme[subsectionpage=progressbar]{metropolis}
  \setbeamertemplate{navigation symbols}{}
  \setbeamertemplate{caption}[numbered]
  \setbeamertemplate{section in toc}[ball unnumbered]
  \setbeamertemplate{subsection in toc}[ball unnumbered]
  \setbeamercovered{transparent}
}

%\usetheme[progressbar=frametitle]{metropolis}
\usepackage{appendixnumberbeamer}

\usepackage[english]{babel}
\usepackage[utf8x]{inputenc}
\usepackage[T1]{fontenc}
\usepackage{url}
\usepackage{times}
\usepackage{listings,xcolor,beramono}
\usepackage{tikz}

% This allows `code` and @code@ to be used inside an lstlisting environment, highlighted with a color box.
% Blatantly stolen from: https://tex.stackexchange.com/a/49309
\makeatletter
\newenvironment{btHighlight}[1][]
{\begingroup\tikzset{bt@Highlight@par/.style={#1}}\begin{lrbox}{\@tempboxa}}
{\end{lrbox}\bt@HL@box[bt@Highlight@par]{\@tempboxa}\endgroup}

\newcommand\btHL[1][]{%
  \begin{btHighlight}[#1]\bgroup\aftergroup\bt@HL@endenv%
}
\def\bt@HL@endenv{%
  \end{btHighlight}%   
  \egroup
}
\newcommand{\bt@HL@box}[2][]{%
  \tikz[#1]{%
    \pgfpathrectangle{\pgfpoint{1pt}{0pt}}{\pgfpoint{\wd #2}{\ht #2}}%
    \pgfusepath{use as bounding box}%
    \node[anchor=base west, fill=orange!30,outer sep=0pt,inner xsep=1pt, inner ysep=0pt, rounded corners=3pt, minimum height=\ht\strutbox+1pt,#1]{\raisebox{1pt}{\strut}\strut\usebox{#2}};
  }%
}
\makeatother

\lstset{language=C++,
  basicstyle=\small\ttfamily,
  keywordstyle=\color{blue}\ttfamily,
  stringstyle=\color{red}\ttfamily,
  commentstyle=\color{magenta}\ttfamily,
  frame=l,
  prebreak=\raisebox{0ex}[0ex][0ex]{\ensuremath{\hookleftarrow}},
  breaklines=true,
  breakatwhitespace=true,
  tabsize=4,
  %escapeinside={@}{@},
  numbers=left,
  xleftmargin=.6cm,
  showstringspaces=false,
  captionpos=b,
  emph={malloc, realloc, calloc, free},
  emphstyle=\color{purple},
  escapeinside={(@}{@)},
  moredelim=**[is][\btHL]{`}{`},
  moredelim=**[is][{\btHL[fill=red!30,draw=yellow,dashed,thin]}]{@}{@},
  moredelim=**[is][{\btHL[fill=blue!20,draw=white,dashed,thin]}]{\$}{\$},
}

\lstdefinestyle{tiny}{
  basicstyle=\scriptsize\ttfamily
}

\usepackage{xspace}

\definecolor{light-gray}{gray}{0.80}
\definecolor{light-yellow}{rgb}{1, 1, 0.60}
\definecolor{light-blue}{rgb}{0.60, 0.80, 0.90}
\definecolor{light-red}{rgb}{1, 0.40, 0.40}

\newcommand{\hl}[2]{\colorbox{#1}{#2}}
\newcommand{\hly}[1]{\hl{light-yellow}{#1}}
\newcommand{\hlg}[1]{\hl{light-gray}{#1}}
\newcommand{\hlb}[1]{\hl{light-blue}{#1}}
\newcommand{\hlr}[1]{\hl{light-red}{#1}}

\newcommand{\todo}[1]{\textbf{\textcolor{red}{TODO: #1}}}

\title{Dangless Malloc}
\subtitle{Mitigating dangling pointer bugs -- thesis presentation}
\date{2018 September 24}
\author{Gábor Kozár}
\institute{VU University Amsterdam, University of Amsterdam}
% \titlegraphic{\hfill\includegraphics[height=1.5cm]{logo.pdf}}

\begin{document}

\maketitle

\begin{frame}{Table of contents}
  \setbeamertemplate{section in toc}[sections numbered]
  %\tableofcontents[hideallsubsections]
  \tableofcontents
\end{frame}

\section{Motivation: dangling pointers}

\begin{frame}[fragile]{Dangling pointers}
    \begin{lstlisting}
int *pmagic = malloc(sizeof(int));
*pmagic = 0xBADF00D;

// ... use pmagic ...
free(pmagic);

// ...
printf("Magic: %d\n", @*pmagic@);
    \end{lstlisting}
\end{frame}

\begin{frame}[fragile]{Dangling pointers and memory re-use}
    \begin{lstlisting}
int *pmagic = malloc(sizeof(int));
*pmagic = 0xBADF00D;

// ... use pmagic ...

free(pmagic);

// ...

// perform another allocation
// memory may be re-used!
char *name = malloc(32);
strcpy(name, "MAGIC!");

// ...
printf("Magic: %d\n", @*pmagic@);
    \end{lstlisting}
\end{frame}

\subsection{As security vulnerability}

\begin{frame}[fragile]{Dangling pointers as security vulnerabilities}
    \begin{minipage}[t]{0.5\linewidth}
        \begin{lstlisting}[style=tiny]
char *x; size_t x_len;

void write_x(char *data) {
  size_t len = strlen(data);
  if ($x_len <= len$) {
    x_len = len + 1;
    x = `realloc(x, x_len)`;
  }
  (@@)@strcpy(x, data)@;
}

void process_x(void) {
  // ...
  (@@)`free(x)`;
}
        \end{lstlisting}
    \end{minipage}
    \begin{minipage}[t]{0.49\linewidth}
        \pause
        \begin{lstlisting}[style=tiny]
struct {
  char p[10];
  (@@)@int is_admin@;
} *y;

void write_y(char *data) {
  if (!y)
    (@@)`y = calloc(sizeof(*y))`;

  if ($strlen(data) >= 10$)
    return;

  strcpy(y->p, data);
  (@@)@process_cmd(y->p, y->is_admin)@;
}
        \end{lstlisting}
    \end{minipage}
    \pause
    \scriptsize
    \lstset{style=tiny}
    \begin{minipage}{0.5\linewidth}
        \begin{enumerate}
            \item<3-> \lstinline[style=tiny]!write_x("dummy 1 + padding")!
            \item<4-> \lstinline[style=tiny]!process_x()!
            \item<5-> \lstinline[style=tiny]!write_y("dummy 2")!
            \item<6-> \lstinline[style=tiny]!write_x("fill buff PWND")!
            \item<7-> \lstinline[style=tiny]!write_y("drop_db")!
        \end{enumerate}
    \end{minipage}
    \begin{minipage}{0.49\linewidth}
        \begin{itemize}
            \item<3-> allocate \lstinline!x!, set \lstinline!x_len! to big enough
            \item<4-> free \lstinline!x!, it is now dangling
            \item<5-> allocate \lstinline!y!, memory is re-used: \lstinline!x = y!
            \item<6-> write to \lstinline!y!, bypassing length check
            \item<7-> enjoy your modified \lstinline!y->is_admin!
        \end{itemize}
    \end{minipage}
\end{frame}

\section{Overview: memory allocation}

\subsection{Normal memory allocation}

\begin{frame}{Normal memory allocation -- step by step}
    \begin{center}
        \includegraphics[width=\linewidth, trim={0 290pt 0 0}, clip]<1>{img/normal_malloc.png}
        \includegraphics[width=\linewidth, trim={0 220pt 0 0}, clip]<2>{img/normal_malloc.png}
        \includegraphics[width=\linewidth, trim={0 150pt 0 0}, clip]<3>{img/normal_malloc.png}
        \includegraphics[width=\linewidth, trim={0 75pt 0 0}, clip]<4>{img/normal_malloc.png}
        \includegraphics[width=\linewidth]<5>{img/normal_malloc.png}
    \end{center}
\end{frame}

\subsection{Dangless allocation -- simplified}

\begin{frame}{Dangless memory allocation -- simplified}
    \begin{center}
        \includegraphics[width=\linewidth, trim={0 302pt 0 0}, clip]<1>{img/dangless_simplified.png}
        \includegraphics[width=\linewidth, trim={0 227pt 0 0}, clip]<2>{img/dangless_simplified.png}
        \includegraphics[width=\linewidth, trim={0 151pt 0 0}, clip]<3>{img/dangless_simplified.png}
        \includegraphics[width=\linewidth, trim={0 77pt 0 0}, clip]<4>{img/dangless_simplified.png}
        \includegraphics[width=\linewidth]<5>{img/dangless_simplified.png}
    \end{center}
\end{frame}

\subsection{Dangless allocation -- the full story}

\begin{frame}{Dangless memory allocation -- full memory view}
    \begin{center}
        \includegraphics[width=\linewidth, trim={0 270pt 0 0}, clip]<1>{img/dangless_virtremap.png}
    \end{center}
\end{frame}

% called virtual alises in the abstract
\begin{frame}{Dangless memory allocation -- virtual remapping}
    \begin{center}
        \includegraphics[width=\linewidth]<1>{img/dangless_virtremap_mappings_xy.png}
        \includegraphics[width=\linewidth]<2>{img/dangless_virtremap_mappings_xyz.png}
    \end{center}
\end{frame}

\begin{frame}{Dangless memory allocation -- step by step}
    \begin{center}
        \includegraphics[width=\linewidth, trim={0 302pt 0 0}, clip]<1>{img/dangless_virtremap.png}
        \includegraphics[width=\linewidth, trim={0 190pt 0 0}, clip]<2>{img/dangless_virtremap.png}
        \includegraphics[width=\linewidth, trim={0 110pt 0 0}, clip]<3>{img/dangless_virtremap.png}
        \includegraphics[width=\linewidth]<4>{img/dangless_virtremap.png}
    \end{center}
\end{frame}

\section{Implementation details}

\subsection{Light-weight virtualization via Dune}

\begin{frame}{Light-weight virtualization via Dune}
    \begin{itemize}
        \item Dune: library and kernel module (based on KVM)
        \item Low-overhead virtualization
        \item Just call \lstinline!dune_init_and_enter()!
        \item<2-> Virtual environment: ring 0 privileges
            \begin{itemize}
                \item \textbf{Manipulate page tables}
                \item Handle interrupts
                \item Handle system calls
            \end{itemize}
        \item<3-> \lstinline!vmcall! to syscall the host kernel
    \end{itemize}
\end{frame}

\subsection{\texttt{vmcall} with remapped pointers}

\begin{frame}[fragile]{\texttt{vmcall} with remapped pointers}
    \begin{itemize}
        \item Virtualization: host (Linux) memory \lstinline|!=| guest (libdune) memory
        \item<2-> Manipulating page tables in guest: no effect on host
        \item<3-> Dangless' virtual aliases: unknown to host
    \end{itemize}

    \begin{onlyenv}<3->
        \begin{lstlisting}
char *`filename = malloc(32)`;
// ...

int fd = open((@@)@filename@, O_CREAT | O_WRONLY);
// ...
        \end{lstlisting}
    \end{onlyenv}
\end{frame}

\begin{frame}[fragile]{Fixing up \texttt{vmcall} arguments}
    \begin{itemize}
        \item \emph{linux-syscallmd}: Python script, parse Linux \emph{syscall.h}
        \item<2-> Patched Dune: \lstinline!vmcall! pre- and post-hooks
        \item<3-> Pre-hook: replace virtual alias pointer arguments with the canonical pointer
    \end{itemize}
    
    \begin{onlyenv}<2->
        \begin{lstlisting}
void dangless_vmcall_prehook(uint64_t *syscallno, uint64_t args[], uint64_t *retaddr);
void dangless_vmcall_posthook(uint64_t result);
        \end{lstlisting}
    \end{onlyenv}
\end{frame}

\begin{frame}[fragile]{Fixing up nested pointers}
    \begin{lstlisting}
struct iovec {
    void  *iov_base; /* Starting address */
    size_t iov_len;  /* Number of bytes */
};

ssize_t readv(int fd, const struct iovec *iov,  int iovcnt);
ssize_t writev(int fd, const struct iovec *iov, int iovcnt);
    \end{lstlisting}

    \pause

    Other examples:
    \begin{itemize}
        \item \lstinline!execve()!: arrays of pointers (\lstinline!argv!, \lstinline!environ!)
        \item \lstinline!recvmsg()!: complex \lstinline!struct msghdr! (incl. \lstinline!struct iovec!)
    \end{itemize}
\end{frame}

\section{Performance evaluation}

\begin{frame}{Performance overhead}
    TODO
\end{frame}

\begin{frame}{Memory overhead}
    TODO
\end{frame}

\section{Conclusion}

\begin{frame}{Conclusion}
    \begin{itemize}
        \item TODO
    \end{itemize}
\end{frame}

\begin{frame}[fragile]{Limitations \& possible future work}
    \begin{itemize}
        \item Not thread-safe (Dune either)
        \item \lstinline!clone()! and \lstinline!fork()! unsupported
        \item Virtual memory does eventually run out (GC)
    \end{itemize}
\end{frame}

\begin{frame}[standout]
    \Huge
    \centering
    Thank you for your attention!
\end{frame}

\end{document}
