\chapter{User Guide}
\label{ch:user-guide}

This chapter is about building, configuring, and using Dangless.
Most of this information in less detail is also described in the repository \path{README} file.

\section{System requirements}

Most requirements are posed by Dune itself:

\begin{itemize}
	\item A 64-bit x86 Linux environment
	\item A relatively recent Intel CPU with VT-x support
	\item Kernel version of 4.4.0 or older
	\item Installed kernel headers for the running kernel
	\item Root (sudo) privileges
	\item Enabled and sufficient number of hugepages (see below)
\end{itemize}

The remaining requirements posed by Dangless itself are fairly usual:

\begin{itemize}
	\item A recent C compiler that supports C11 and the GNU extensions (either GCC or Clang will work)
	\item Python 3.6.1 or newer
	\item CMake 3.5.2 or newer
\end{itemize}

\subsection{Hugepages}

Besides the above, Dune requires some 2 MB hugepages to be available during initialization for setting up its safe stacks. It will also try to use huge pages to acquire memory for the guest's page allocator, although it will gracefully fall back if there are not enough huge pages available.

To make sure that some huge pages remain available, it is recommended to limit or disable transparent hugepages by setting \path{/sys/kernel/mm/transparent_hugepage/enabled} to \emph{madvise} or \emph{never} (you will need to use \path{su} if you want to change it).

Then, you can check the number of huge pages available:

\begin{lstlisting}[breaklines, language=, style=]
$ cat /proc/meminfo | grep Huge
AnonHugePages:     49152 kB
HugePages_Total:     512
HugePages_Free:      512
HugePages_Rsvd:        0
HugePages_Surp:        0
Hugepagesize:       2048 kB
\end{lstlisting}

In my tests, it appears that at minimum \textbf{71} free huge pages are required to satisfy Dune, although it is not quite clear to me as to why: by default for 2 safe stacks of size 2 MB each, we should only need 2 hugepages.

You can dedicate more huge pages by modifying \path{/proc/sys/vm/nr_hugepages} (again, you will need to use \texttt{su} to do so), or by executing:

\begin{lstlisting}[breaklines, language=, style=]
sudo sysctl -w vm.nr_hugepages=<NUM>
\end{lstlisting}

... where \texttt{<NUM>} should be replaced by the desired number, of course.

When there is not a sufficient number of huge pages available, Dangless will fail while trying to enter Dune mode, and you will see output much like this:

\begin{lstlisting}[breaklines, language=, style=]
dune: failed to mmap() hugepage of size 2097152 for safe stack 0
dune: setup_safe_stack() failed
dune: create_percpu() failed
Dangless: failed to enter Dune mode: Cannot allocate memory
\end{lstlisting}

\section{Building and configuring}

The full Dangless source code is available on GitHub at \url{https://github.com/shdnx/dangless-malloc}. After cloning, you will have to start by setting up its dependencies (such as Dune) which are registered as git submodules in the \path{vendor} folder:

\begin{lstlisting}[breaklines, language=bash, style=]
git submodule init
git submodule update
\end{lstlisting}

Then we have to apply the Dune patches as described in Section~\ref{sec:bg-dune} and build it:

\begin{lstlisting}[breaklines, language=bash, style=]
cd vendor/dune-ix

# patch dune, so that the physical page metadata is accessible inside the guest, allowing us to e.g. mess with the pagetables
git apply ../dune-ix-guestppages.patch

# patch dune, so that we can register a prehook function to run before system calls are passed to the host kernel
git apply ../dune-ix-vmcallprehook.patch

# patch dune, so that it does not kill the process with SIGTERM when handling the exit_group syscall - this causes runs to be registered as failures when they succeeded
git apply ../dune-ix-nosigterm.patch

# need sudo, because it is building a kernel module
sudo make
\end{lstlisting}

Now configure and build Dangless using CMake:

\begin{lstlisting}[breaklines, language=bash, style=]
cd ../../sources

mkdir build
cd build

# you can specify your configuration options here, or e.g. use ninja (-GNinja) instead of make
cmake \
	-D CMAKE_BUILD_TYPE=Debug \
	-D OVERRIDE_SYMBOLS=ON \
	-D REGISTER_PREINIT=ON \
	-D COLLECT_STATISTICS=OFF \
	..

cmake --build .
\end{lstlisting}

You should be able to see \path{libdangless_malloc.a} and \path{dangless_user.make} afterwards in the build directory.

You can see what configuration options were used to build Dangless by listing the CMake cache:

\begin{lstlisting}[breaklines, language=C, style=]
$ cmake -LH
-- Cache values
// Whether to allow dangless to gracefully handle running out of virtual memory and continue operating as a proxy to the underlying memory allocator.
ALLOW_SYSMALLOC_FALLBACK:BOOL=ON

// Whether Dangless should automatically dedicate any unused PML4 pagetable entries (large unused virtual memory regions) for its virtual memory allocator. If disabled, user code will have to call dangless_dedicate_vmem().
AUTODEDICATE_PML4ES:BOOL=ON

// Choose the type of build, options are: None(CMAKE_CXX_FLAGS or CMAKE_C_FLAGS used) Debug Release RelWithDebInfo MinSizeRel.
CMAKE_BUILD_TYPE:STRING=Debug

// Install path prefix, prepended onto install directories.
CMAKE_INSTALL_PREFIX:PATH=/usr/local

// Whether to collect statistics during runtime about Dangless usage. If enabled, statistics are printed after every run to stderr. These are only for local developer use and are not uploaded anywhere.
COLLECT_STATISTICS:BOOL=OFF

// Debug mode for dangless_malloc.c
DEBUG_DGLMALLOC:BOOL=OFF

// Debug mode for vmcall_fixup.c
DEBUG_DUNE_VMCALL_FIXUP:BOOL=OFF
...
\end{lstlisting}

You can also use a CMake GUI such CCMake~\cite{ccmake-website}, or check the main CMake file (\path{sources/CMakeLists.txt}) for the list of available configuration options, their description and default values.

\section{API overview}

Dangless is a Linux static library \path{libdangless_malloc.a} that can be linked to any application during build time. It defines a set of functions for allocating and deallocating memory:

\begin{lstlisting}
// sources/include/dangless/dangless_malloc.h

void *dangless_malloc(size_t sz) __attribute__((malloc));
void *dangless_calloc(size_t num, size_t size) __attribute__((malloc));
void *dangless_realloc(void *p, size_t new_size);
int dangless_posix_memalign(void **pp, size_t align, size_t size);
void dangless_free(void *p);
\end{lstlisting}

These functions have the exact same signature and behaviour as their standard counterparts \lstinline!malloc()!, \lstinline!calloc()!, and \lstinline!free()!. In fact, because the GNU C Library defines these standard functions as weak symbols~\cite{glibc-malloc-is-weak}, Dangless provides an option (\lstinline!OVERRIDE_SYMBOLS!) to override the symbols with its own implementation, enabling the user code to perform memory management without even being aware that it is using Dangless in the background.

Besides the above, Dangless defines a few more functions, out of which the following two are important.

\begin{lstlisting}
void dangless_init(void);
\end{lstlisting}

Initializes Dangless as described in Section~\ref{sec:dangless-init}. Whether or not this function is called automatically during application start-up is controlled by the \lstinline!REGISTER_PREINIT! option, defaulting to On.

\begin{lstlisting}
int dangless_dedicate_vmem(void *start, void *end);
\end{lstlisting}

Dedicates a memory region to Dangless's virtual memory allocator, as described in Section~ref{sec:dangless-alloc-virtmem}.
Whether or not any dedication happens automatically is controlled by the \lstinline!AUTO_DEDICATE_PML4ES! option.

\section{Integrating into existing applications}

Dangless can be integrated into an application which relies, directly or indirectly, the C memory management API. To do this:
\begin{itemize}
	\item link to \path{libdangless_malloc.a} using the \texttt{--whole-library} option
	\item link to \path{libdune.a}
	\item link to \path{libdl} to allow Dangless to resolve the symbols for the system allocator
	\item disable building a position-independent executable (PIE) using \texttt{-no-pie}, due to the limitation of Dune
\end{itemize}

When building, Dangless generates a \path{dangless_user.make} file that makes this easier to do with Makefile-based build systems.
