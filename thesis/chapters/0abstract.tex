\abstract{
	Manual memory management required in programming languages like C and C++ has its advantages, but comes at a significant cost in complexity, frequently leading to bugs and security vulnerabilities. One such example is temporal memory errors, whereby an object is accessed after it has been deallocated, through a pointer (or C++ reference) that outlived it: a dangling pointer. Such bugs, even after decades of research and development both in tooling and programming languages, are still extremely common today.

	We propose a new solution, called Dangless, which protects against temporal memory errors by ensuring that any memory accesses through dangling pointers are detected immediately and cause the application to be terminated. This is achieved by maintaining a unique virtual alias for each individual allocation, similarly to the shadow pages in previous work such as Electric Fence~\cite{electric-fence} and Oscar~\cite{oscar2017}. Dangless iterates on these existing techniques by significantly improving performance by running the process in a light-weight virtual environment, giving us direct access to the page tables and allowing us to remap virtual pages and modify page protection flags without having to pay the overhead of system calls.
	
	We have evaluated performance on the SPEC2006 benchmarking suite and have found a geometric mean of 5.7\% runtime performance overhead and 8.2\% physical memory overhead. This makes Dangless one of the best-performing tools for preventing temporal memory errors, although its practical usefulness is reduced by not offering protection against a more widespread set of memory errors.
}
