Abstract

Manual memory management required in programming languages like C and C++ has its advantages, but comes at a cost in complexity, frequently leading to bugs and security vulnerabilities. One such example is temporal memory errors, whereby an object or memory region is accessed after it has been deallocated. The pointer through which this access occurs is said to be dangling.

Our solution, Dangless, protects against such bugs by ensuring that any references through dangling pointers are caught immediately. This is done by maintaining a unique virtual alias for each individual allocation. We do this efficiently by running the process in a light-weight virtual environment, where the allocator can directly modify the page tables.

We have evaluated performance on the SPEC2006 benchmarking suite, and on a subset of the benchmarks have found a geometric mean of 3.5\% runtime performance overhead and 406\% memory overhead. This makes this solution very efficient in performance - comparable to other state-of-the-art solutions - but the high memory overhead limits its usability in practice.
